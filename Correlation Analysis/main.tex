\documentclass[11pt,twoside,a4paper]{article}
\usepackage{amssymb}
\usepackage{tensor}
\usepackage{pythontex}
\usepackage{graphicx}
\usepackage{subcaption}
\usepackage{url}
\usepackage{placeins} 
\usepackage{multirow}
\begin{document}
\pyc{import python.base as base, python.code as code}
\title{Correlation analysis}
\author{Thomas Bütikofer}
\date{\today}
\maketitle
\begin{abstract}
Your abstract goes here...
\end{abstract}
\section{Introduction}
%\py{2+5}
This article analyses the collective dynamics for gains/losses of securities traded in a market over a time period.
The series of closing quotes $q$ for a security $s$, traded on market $m$ during the time interval $\mathbb{T}$ with $T+1$ trading days are denoted by
\begin{equation}
q^{(m)}_{s}:\mathbb{T}\rightarrow\mathbb{R}\quad.
\label{eqDefQuoteSeries}
\end{equation}
Let $\mathbb{S}^{(m)}_\mathbb{T}$ be the set of $S$ securities which are all traded both, at the first and last trading day of a market $m$ within the interval $\mathbb{T}$.\\
By introducing an arbitrary order into $\mathbb{S}^{(m)}_\mathbb{T}$, the function
\begin{eqnarray}
Q^{(m)}_{\mathbb{T}}:\mathbb{S}\times\mathbb{T}&\rightarrow&\mathbb{R}\\
(s,t)&\mapsto&q^{(m)}_s(t)
\end{eqnarray}
defines a matrix $Q^{(m)}_{\mathbb{T}}$ for $S$ securities with $T+1$ quotes each:
\begin{equation}
(Q^{(m)}_{\mathbb{T}})_{st}\equiv Q^{(m)}_{\mathbb{T}}(s,t)\quad.
\end{equation}
Instead of analyzing absolute changes in a security's price, this article will use the series $\Delta^{(m)}_{\mathbb{T}}$ of relative changes, normalized by a logarithm 
\begin{equation}
\left(\Delta^{(m)}_{\mathbb{T}}\right)_{st} \equiv \log\left[\frac{Q^{(m)}_{\mathbb{T}}(s,t)}{Q^{(m)}_{\mathbb{T}}(s,t-1)}\right]\quad t\in\{1\ldots T\}\quad.
\end{equation}
$\Delta^{(m)}_{\mathbb{T}}$ is a matrix of dimension $T\times N$.\\

This transformation intends to focus the following analysis on relative changes of a security's quote and put the same attention on gains and losses.\\
The rest of this article will on most occasions omit the indication of the market and the relevant time period and just use $\Delta_s$ for the time series of a specific security $s$, and $\Delta$ for the set of the time series of all considered securities.\\
The average over all securities will be denoted as
\begin{equation}
\langle\Delta(t)\rangle_\mathbb{S}\equiv\frac{1}{S}\sum_{s\in \mathbb{S}}\Delta_s(t)\quad,
\end{equation}
while the average over time will be written as
\begin{equation}
\langle\Delta_s\rangle_\mathbb{T}\equiv\frac{1}{T}\sum_{t\in \mathbb{T}}\Delta_s(t)\quad.
\end{equation}
In order to determine what information can be gleaned from the principal component and market stress analysis, both methods will be applied to daily quotes from different markets and the results presented in the following section.
\section{Methods}
\subsection{Data preparation}
For less volatile securities, gaps may occur in their time series of day closing quotes on trading days without a single trade. Such gaps are filled with the last known value of the preceding trading day.\\
Market holidays are regarded as non-existent. I.e. trading days are considered as subsequent, even if they are separated by one or multiple market holidays. The motivation being that without trades, there is no market.
\subsection{Market stress analysis}
The Pearson correlation between security quote time series over a time interval $\mathbb{T}$ can be used to quantify the similarity between a pair of security quote series.
\begin{equation}
\left(C_\mathbb{S}\right)_{s_1s_2}=\frac{T}{T-1}\frac{\langle\left(\Delta_{s_1}-\langle\Delta_{s_1}\rangle_\mathbb{T}\right)\left(\Delta_{s_2}-\langle\Delta_{s_2}\rangle_\mathbb{T}\right)\rangle_\mathbb{T}}{\sigma(\Delta_{s_1})\sigma(\Delta_{s_2})}
\end{equation}
with $\sigma_s$ being the standard deviation over the time series $\Delta_s$:
\begin{equation}
\sigma_s \equiv \sqrt{\frac{T}{T-1}\langle\left(\Delta_{s}-\langle\Delta_{s}\rangle_\mathbb{T}\right)^2\rangle_\mathbb{T}}
\end{equation}
This measure normalizes the gain/loss amplitudes of $\Delta$ and is suitable to identify securities with comparable trends without regard to the relative changes.\\
This article defines the market stress matrix $M_\mathbb{S}$ as:
\begin{equation}
M_\mathbb{S}\equiv C_\mathbb{S} - \textrm{1}_\mathbb{S}\quad.
\end{equation}
The market stress matrix $M_\mathbb{S}$ is a symmetric with diagonal elements all equal to $0$.\\
Collective price movements during a trading interval indicate that the securities prices are not assessed individually by the market, but that a collection of securities react to some kind of market stress\footnote{Note that $M_\mathbb{S}$ also resemble a viscous stress tensor of a perfectly fluid material in $S$ dimensions.}.\\
With this measure, $M_\mathbb{S}$ quantifies for each security $s$ the experienced stress $f_s$ during the time period $\mathbb{T}$.
\begin{equation}
f_s\equiv (M_\mathbb{S}^\top M_\mathbb{S})_{ss}
\end{equation}
The security's stress $f_s$ is an emergent property of an individual security, reacting on the market dynamics.\\

As a measure of the average market stress this article will use
\begin{equation}
f^{(m)}_\mathbb{T} \equiv \langle f_s\rangle=\frac{1}{S}\sum_{s\in\mathbb{S}}f_s=\frac{1}{S}Tr\left(M_\mathbb{S}^\top M_\mathbb{S}\right)\quad.
\end{equation}
\begin{pycode}
securities = 100
change_range = (-0.5, 0.5)
filename = 'market_norm_stress_uniform'
#filename = code.simulate_random_stress(trading_days_list, securities, (), change_range)
param_fit,  param_error= base.plot_normal_stress(filename)
\end{pycode}
A market can be assigned an intrinsic average stress, by assuming completely random quote fluctuations $\Delta_\mathbb{T}$. This intrinsic stress can be used as a baseline to compare against the effective market stress.\\
By simulating quote fluctuations $\Delta_\mathbb{T}$ with a uniform probability distribution between $\left[-\Delta,\Delta\right]$, the average market stress shows a dependency on the length of the time period $\mathbb{T}$. This relationship can be approximated by
\begin{equation}
f^0_\mathbb{T} \approx \alpha T^\beta\qquad\alpha=\py{"{:0.2f}".format(param_fit[0])}\pm\py{"{:0.2f}".format(param_error[0])}\quad\beta=\py{"{:0.2f}".format(param_fit[1])}\pm\py{"{:0.2f}".format(param_error[1])}\quad.
\label{eqInstrinsicMarketStress}
\end{equation}
\begin{figure}
\centering
	% trim={<left> <lower> <right> <upper>}
\pyc{print(r'\includegraphics[width=\columnwidth,trim={{40 15 50 50}},clip]{{{{"img/{}"}}}}'.format(filename))}
\caption{Dependency of the intrinsic market stress $f^0_\mathbb{T}$ on $T$ on a logarithmic scale. Circles show simulated average market stresses for a market with \py{securities} securities and a uniform distribution between the interval $\left[\py{change_range[0]}, \py{change_range[1]}\right]$ for $\Delta_\mathbb{T}$. The dotted line indicates the approximation  (\ref{eqInstrinsicMarketStress})}.
\label{figIntrinsicMarketStress}
\end{figure}
\FloatBarrier
\subsection{Principal securities}
At times with high market stress $f^{(m)}_\mathbb{T}$ price movements of many securities synchronize. This article defines principal securities as the set of securities whose price movements always synchronize at times of high market stress. In this sense, principal securities represent the general price dynamics of the whole market in times of high market stress.\\
The search for such principal securities uses the stress matrix $M_\mathbb{S}$ at times of maximal $f^{(m)}_\mathbb{T}$ and employs the K-means algorithm to group all securities into 4 clusters with similar stress vectors. If, for a list of high stress days, the same securities always end up in the cluster with highest stress average, they are considered candidates for principal securities of the respective market.\\

The existence of principal securities should not be taken for granted.
\FloatBarrier
\pagebreak
\section{Results}
\begin{pycode}
market_events = base.load_list('market_stress_events')
\end{pycode}
\subsection{Swiss stock exchange SWX}
\begin{pycode}
market_code = 'SWX'
date_from = '01.01.2000'
date_to = '31.12.2018'
zoom_date_from = '01.01.2016'
zoom_date_to = '31.12.2016'
series_length = 20

isin_count = base.market_info(market_code, date_from, date_to)

filename_market_full_stress = 'market_momentum_SWX_20000101-20181231'
#filename_market_full_stress = code.calc_market_momentum(market_code,date_from,date_to, series_lengths=[series_length], step_days=7, offset=0)
base.plot_time_series(filename_market_full_stress, legends=[''], events=market_events)
filename_market_zoom_stress = 'market_momentum_SWX_20160101-20161231'
#filename_market_zoom_stress = code.calc_market_momentum(market_code,zoom_date_from,zoom_date_to, series_lengths=[series_length], step_days=1, offset=0)
base.plot_time_series(filename_market_zoom_stress, legends=[''], events=market_events)
\end{pycode}
For the SIX Swiss exchange (\py{market_code}) the daily changes in quotes of \py{isin_count} securities with CH... ISIN codes over the time interval \py{date_from} - \py{date_to} have been analyzed.\\
During this time interval the market experienced several external economic and/or political events with potential effects on the traded security's quotes. Most notable among these has certainly been the decisions of the Swiss national bank to cap the CHF-EUR exchange rate and subsequent removal of the cap.\\

The market stress $f_\mathbb{T}$ for \py{market_code} with time interval length $T=\py{series_length}$ is shown in figure \ref{figSWXstress} (top). Over a background noise of $0.25 < f_\mathbb{T} < 0.30$, distinctive peaks in market stress exist. See table \ref{tab_marketevents} for a detailed list of events with market stress above $0.35$.\\

Zooming into the period \py{zoom_date_from} - \py{zoom_date_to}, which includes the Brexit referendum in the UK, sheds some light on the dynamics linking this event to the market stress (figure \ref{figSWXstress}, bottom). The market stress increases sharply in the days following the publication of the referendum's results, followed by two even more pronounced peaks. The width of the elevated stress levels above 0.30 corresponds to the resolution of $T=\py{series_length}$ of this analysis.
\\
\begin{figure}
\centering
\begin{subfigure}[b]{\textwidth}
	% trim={<left> <lower> <right> <upper>}
	\pyc{print(r'\includegraphics[width=\columnwidth,trim={{40 40 50 50}},clip]{{{{"img/{}"}}}}'.format(filename_market_full_stress))}
\end{subfigure}
\begin{subfigure}[b]{\textwidth}
	\pyc{print(r'\includegraphics[width=\columnwidth,trim={{40 40 50 50}},clip]{{{{"img/{}"}}}}'.format(filename_market_zoom_stress))}
\end{subfigure}
\caption{Market stress for \py{market_code} over the interval length $T=\py{series_length}$. The horizontal axis indicates the last day of the time period resulting in the market stress $f_\mathbb{T}$. Notable events from table \ref{tab_marketevents} are indicated as dotted vertical lines.\\
Top: Market stress over the whole analyzed time period. Bottom: Enlarged section containing the Brexit referendum.
}
\label{figSWXstress}
\end{figure}

While figure \ref{figSWXstress} indicates dates with high average correlation between security quote series, figure \ref{figSWXstressmap} gives an indication how many of the traded securities can follow similar trends. Although on trading days with increased market stress more  quotes behave similarly, the events affect by no means all securities equally strong. By using the K-means algorithm to group all securities into 4 clusters with similar stress vectors, the most affected securities for each event 
  
\begin{pycode}
market_code = 'SWX'
date_from =  '01.01.2013'
date_to = '31.12.2018'
series_length = 20
#filename = code.find_base_securities(market_code, market_events, date_from, date_to, series_length)
filename = 'market_base_SWX_20130101-20181231-20'
security_list = base.load_list(filename)
\end{pycode}
\begin{table}
\centering
\begin{tabular}{l|l}
\pyc{base.latex_table(security_list)}
\end{tabular}
\caption{Principal securities for \py{market_code}. The listed securities exhibited the general market price movements in all stress events (see table \ref{tab_marketevents}) during the time period \py{date_from} - \py{date_to}.}
\label{tabSWXstressedSecurities}
\end{table}
\begin{pycode}
market_code = 'SWX'
date_low = '18.03.2017'
date_high = '17.01.2015'
series_length = 20
cluster_num = 1
#filename_low = code.calc_market_stress(market_code, date_low, series_length, cluster_num)
filename_low = 'market_stress_SWX_20170318-20'
base.plot_matrix(filename_low, (-1,1))
#filename_high = code.calc_market_stress(market_code, date_high, series_length, cluster_num)
filename_high = 'market_stress_SWX_20150117-20'
base.plot_matrix(filename_high, (-1,1))
\end{pycode}
\begin{figure}
\centering
\begin{subfigure}[b]{0.48\textwidth}
	% trim={<left> <lower> <right> <upper>}
	\pyc{print(r'\includegraphics[width=\textwidth,trim={{55 50 0 60}},clip]{{{{"img/{}"}}}}'.format(filename_low))}
\end{subfigure}
~
\begin{subfigure}[b]{0.48\textwidth}
	\pyc{print(r'\includegraphics[width=\textwidth,trim={{55 50 0 60}},clip]{{{{"img/{}"}}}}'.format(filename_high))}
\end{subfigure}
\caption{Heat map of market stress $M_\mathbb{S}$ between pairs of traded securities at \py{market_code}. Securities are ordered by decreasing stress $(f_\mathbb{T})_s$.\\
Left: $\py{date_low}$ as an example of a market under minimal stress. Right: $\py{date_high}$, after the announcement of the Swiss national bank to decouple the Swiss franc from the Euro caused considerable market stress.}
\label{figSWXstressmap}
\end{figure}
\FloatBarrier
\subsection{German stock exchange XETR}
\begin{pycode}
market_code = 'XETR'
date_from = '01.01.2003'
date_to = '31.12.2018'
series_length = 20

isin_count = base.market_info(market_code, date_from, date_to)

filename_market_full_stress = 'market_momentum_XTR_20030101-20181231'
#filename_market_full_stress = code.calc_market_momentum(market_code,date_from,date_to, series_lengths=[series_length], step_days=7, offset=0)
base.plot_time_series(filename_market_full_stress, legends=[''], events=market_events)
\end{pycode}
For the German stock exchange Xetra (\py{market_code}) the daily changes in quotes of \py{isin_count} securities with DE... ISIN codes over the time interval \py{date_from} - \py{date_to} have been analyzed.\\
With the disjoint sets of traded securities for SWX and XETR, the analysis of the German market is intended to suggest common features between the two geographically close markets and identify potential properties specific two one of the two markets.\\

The market stress $f_\mathbb{T}$ for \py{market_code} with time interval length $T=\py{series_length}$ is shown in figure \ref{figXETRstress} (top). The range of market stresses and background noise between $0.25 < f_\mathbb{T} < 0.30$ is comparable to SWX. While several stress peaks between the two markets coincide, there are some differences (see table \ref{tab_marketevents}).\\

\begin{figure}
\centering
	% trim={<left> <lower> <right> <upper>}
\pyc{print(r'\includegraphics[width=\columnwidth,trim={{40 40 50 50}},clip]{{{{"img/{}"}}}}'.format(filename_market_full_stress))}
\caption{Market stress for \py{market_code} over the interval length $T=\py{series_length}$. The horizontal axis indicates the last day of the time period resulting in the market stress $f_\mathbb{T}$. Notable events from table \ref{tab_marketevents} are indicated as dotted vertical lines.}
\label{figXETRstress}
\end{figure}
\begin{pycode}
market_code = 'XETR'
date_from =  '01.01.2013'
date_to = '31.12.2018'
series_length = 20
#filename = code.find_base_securities(market_code, market_events, date_from, date_to, series_length)
filename = 'market_base_XETR_20130101-20181231-20'
security_list = base.load_list(filename)
\end{pycode}
\begin{table}
\centering
\begin{tabular}{l|l}
\pyc{base.latex_table(security_list)}
\end{tabular}
\caption{Principal securities for \py{market_code}. The listed securities exhibited the general market price movements in all stress events (see table \ref{tab_marketevents}) during the time period \py{date_from} - \py{date_to}.}
\label{tabXETRstressedSecurities}
\end{table}
\begin{pycode}
market_code = 'XETR'
date_low = '27.09.2017'
date_high = '15.09.2015'
series_length = 20
cluster_num = 1
#filename_low = code.calc_market_stress(market_code, date_low, series_length, cluster_num)
filename_low = 'market_stress_XETR_20170927-20'
base.plot_matrix(filename_low, (-1,1))
#filename_high = code.calc_market_stress(market_code, date_high, series_length, cluster_num)
filename_high = 'market_stress_XETR_20150915-20'
base.plot_matrix(filename_high, (-1,1))
\end{pycode}
\begin{figure}
\centering
\begin{subfigure}[b]{0.48\textwidth}
	% trim={<left> <lower> <right> <upper>}
	\pyc{print(r'\includegraphics[width=\textwidth,trim={{55 50 0 60}},clip]{{{{"img/{}"}}}}'.format(filename_low))}
\end{subfigure}
~
\begin{subfigure}[b]{0.48\textwidth}
	\pyc{print(r'\includegraphics[width=\textwidth,trim={{55 50 0 60}},clip]{{{{"img/{}"}}}}'.format(filename_high))}
\end{subfigure}
\caption{Heat map of market stress $M_\mathbb{S}$ between pairs of traded securities at \py{market_code}. Securities are ordered by decreasing stress $(f_\mathbb{T})_s$.\\
Left: $\py{date_low}$ as an example of a market under minimal stress. Right: $\py{date_high}$.}
\label{figSWXstressmap}
\end{figure}
\FloatBarrier
\newpage
\subsection{Crypto currencies}
\begin{pycode}
market_code = 'CRYP'
date_from = '01.01.2015'
date_to = '31.12.2018'
series_length = 20

isin_count = base.market_info(market_code, date_from, date_to)

filename_market_full_stress = 'market_stress_CRYP_20150101-20181231'
#filename_market_full_stress = code.calc_market_momentum(market_code,date_from,date_to, series_lengths=[series_length], step_days=7, offset=0)
base.plot_time_series(filename_market_full_stress, legends=[''], events=market_events)
\end{pycode}
For the crypto currency data, daily quotes from \url{www.cryptocurrencychart.com} for \py{isin_count} currencies over the time interval \py{date_from} - \py{date_to} have been analyzed.\\
Although all analyzed crypto currency data sets were obtained from \url{www.cryptocurrencychart.com}, the trading platforms which produced the quotes are unknown. The results of the crypto currency analysis therefore do not reflect the experienced stress of a single market, but rather of the global crypto trades.
\begin{figure}
\centering
	% trim={<left> <lower> <right> <upper>}
\pyc{print(r'\includegraphics[width=\columnwidth,trim={{40 40 50 50}},clip]{{{{"img/{}"}}}}'.format(filename_market_full_stress))}
\caption{Market stress for crypto currencies over the interval length $T=\py{series_length}$. The horizontal axis indicates the last day of the time period resulting in the market stress $f_\mathbb{T}$. Notable events from table \ref{tab_marketevents} are indicated as dotted vertical lines.}
\label{figXETRstress}
\end{figure}

\begin{pycode}
market_code = 'CRYP'
date_low = '29.10.2017'
date_high = '20.02.2018'
series_length = 20
cluster_num = 1
#filename_low = code.calc_market_stress(market_code, date_low, series_length, cluster_num)
filename_low = 'market_stress_CRYP_20171029-20'
base.plot_matrix(filename_low, (-1,1))
#filename_high = code.calc_market_stress(market_code, date_high, series_length, cluster_num)
filename_high = 'market_stress_CRYP_20180220-20'
base.plot_matrix(filename_high, (-1,1))
\end{pycode}
\begin{figure}
\centering
\begin{subfigure}[b]{0.48\textwidth}
	% trim={<left> <lower> <right> <upper>}
	\pyc{print(r'\includegraphics[width=\textwidth,trim={{55 50 0 60}},clip]{{{{"img/{}"}}}}'.format(filename_low))}
\end{subfigure}
~
\begin{subfigure}[b]{0.48\textwidth}
	\pyc{print(r'\includegraphics[width=\textwidth,trim={{55 50 0 60}},clip]{{{{"img/{}"}}}}'.format(filename_high))}
\end{subfigure}
\caption{Heat map of market stress $M_\mathbb{S}$ between pairs of crypto currencies. Currencies are ordered by decreasing stress $(f_\mathbb{T})_s$.\\
Left: $\py{date_low}$ as an example of a market under minimal stress. Right: $\py{date_high}$.}
\label{figSWXstressmap}
\end{figure}
\FloatBarrier
\section{Discussion}
\begin{table}
\begin{tabular}{c|c|c|c|l}
\# & SWX & XETR & \multicolumn{2}{|c}{Event}\\
\hline
\multirow{2}{*}{1} & 13.06.2006 		& 15.06.2006     & \multirow{2}{*}{14.06.2006} & \multirow{2}{0.33\linewidth}{}\\
							 & 05.06.-03.07.	& 30.05.-05.07. & &\\
\hline
\multirow{2}{*}{2} & 26.03.2007 		& 24.03.2007     & \multirow{2}{*}{25.03.2007} & \multirow{2}{0.33\linewidth}{}\\
							 & 14.03.-26.03.	& 08.03.-26.03. & &\\
\hline
\multirow{2}{*}{3} & 27.08.2007 		&      & \multirow{2}{*}{27.08.2007} & \multirow{2}{0.33\linewidth}{}\\
							 & 16.08.-12.09.	& & &\\
\hline
\multirow{2}{*}{4} & 16.02.2008 		& 16.02.2008     & \multirow{2}{*}{16.02.2008} & \multirow{2}{0.33\linewidth}{US mortgage crisis, AIG}\\
							 & 24.01.-17.02.	& 04.02.-17.02. & &\\
\hline
\multirow{2}{*}{5} & 15.10.2008 		& 15.10.2008     & \multirow{2}{*}{15.10.2008} & \multirow{2}{0.33\linewidth}{Banking crisis, national bank interventions}\\
							 & 08.10.-11.11.	& 08.10.-09.11. & &\\
\hline
\multirow{2}{*}{6} & 03.06.2010 		& 30.05.2010     & \multirow{2}{*}{01.06.2010} & \multirow{2}{0.33\linewidth}{European sovereign debt crisis: Greece/Spain}\\
							 & 10.05.-14.06.	& 10.05.-15.06. & &\\
\hline
\multirow{2}{*}{7} & & 22.03.2011     & \multirow{2}{*}{22.03.2011} & \multirow{2}{0.33\linewidth}{}\\
							 & & 15.03.-11.04. & &\\
\hline
\multirow{2}{*}{8} & 15.08.2011 		& 14.09.2011  & \multirow{2}{*}{01.09.2011} & \multirow{2}{0.33\linewidth}{06.09.2011:  CHF-EUR cap}\\
							 & 02.08.-02.10.	& 08.08.-19.10. & &\\
\hline
\multirow{2}{*}{9} & 30.06.2013 		& & \multirow{2}{*}{30.06.2013} & \multirow{2}{0.33\linewidth}{}\\
							 & 24.06.-30.06.	& & &\\
\hline
\multirow{2}{*}{10} & 17.01.2015 		& & \multirow{2}{*}{17.01.2015} & \multirow{2}{0.33\linewidth}{15.01.2015: End CHF-EUR cap}\\
							 & 15.01.-12.02.	& & &\\
\hline
\multirow{2}{*}{11} & 27.08.2015 		& 15.09.2015     & \multirow{2}{*}{06.09.2015} & \multirow{2}{0.33\linewidth}{}\\
							 & 24.08.-21.09.	& 21.08.-21.09. & &\\
\hline
\multirow{2}{*}{12} & 23.02.2016 		& 15.02.2016     & \multirow{2}{*}{19.02.2016} & \multirow{2}{0.33\linewidth}{}\\
							 & 22.01.-07.03.	& 20.01.-06.03. & &\\
\hline
\multirow{2}{*}{13} & 13.07.2016 		& 14.07.2016     & \multirow{2}{*}{13.07.2016} & \multirow{2}{0.33\linewidth}{23.06.2016: Brexit referendum}\\
							 & 28.06.-24.07.	& 13.07.-15.07. & &\\
\hline
\multirow{2}{*}{14} & 16.06.2017 		& & \multirow{2}{*}{16.06.2017} & \multirow{2}{0.33\linewidth}{}\\
							 & 13.06.-16.06.	& & &\\
\hline
\multirow{2}{*}{15} & 28.02.2018 		& 02.03.2018     & \multirow{2}{*}{01.03.2018} & \multirow{2}{0.33\linewidth}{}\\
							 & 21.02.-06.03.	& 21.02.-02.03. & &\\
\hline
\end{tabular}
\caption{Periods of elevated (above 0.35) market stress and the day of maximal stress for SWX and XETR. Events indicate average day of maximal stress and possible trigger events and  for the two markets.}
\label{tab_marketevents}
\end{table}
\FloatBarrier
\section{Conclusion}

\bibliography{references}{}
\bibliographystyle{plain}
\end{document}
\end{document}
